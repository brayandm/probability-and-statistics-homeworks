\section*{Problem 3.4}
\addcontentsline{toc}{section}{Problem 3.4}

\textbf{Statement:}

Consider the random uniform number generator \texttt{rand} \textbf{with
    only 3 digits precision}, i.e., we consider equiprobable values
\( 0 \leq x_k \leq 1 \) with only three digits after the decimal
point. Assume we get from it the sequence of values \( x_1, x_2, x_3, x_4 \).

\begin{enumerate}
    \item Find the probability that \( \max(x_1, x_4) < \frac{1}{3} \).
    \item Find the probability that \( \min(x_1, x_2) < \frac{1}{3} \).
    \item Find the probability that \( \max(x_1, x_3) \) is in \( \left[ \frac{1}{4}, \frac{3}{4} \right] \).
    \item Find the probability that exactly two random values out of four are in \( \left[0, \frac{1}{3} \right] \).
\end{enumerate}

\noindent\textbf{Solution:}

\textbf{1)} Let's notice that $x_1$ and $x_4$ are independent
random variables, and they are both uniformly distributed in $[0,1]$
with only three digits after the decimal point.

So we have that $\max(x_1, x_4)$ should be less than $\frac{1}{3}$.
Then $x_1$ and $x_4$ should be both less than $\frac{1}{3}$, because
if one of them is greater or equal than $\frac{1}{3}$, then the maximum will
be greater or equal than $\frac{1}{3}$.

Then because $x_1$ and $x_4$ are independent, we have can calculate
sepately the probability that $x_1$ is less than $\frac{1}{3}$ and the
probability that $x_4$ is less than $\frac{1}{3}$, and then multiply them
together.

In other words, we have that:

\begin{equation*}
    \begin{split}
        P(\max(x_1, x_4) < \frac{1}{3}) & = P(x_1 < \frac{1}{3} \cap x_4 < \frac{1}{3})     \\
                                        & = P(x_1 < \frac{1}{3}) \cdot P(x_4 < \frac{1}{3}) \\
    \end{split}
\end{equation*}

To calculate $P(x_1 < \frac{1}{3})$ we can notice that $x_1$ is
uniformly distributed in $[0,1]$ with only three digits after the
decimal point. Then the total number of possible values for $x_1$ is
$1001$ (because we can make a bijection between a number with three
digits after the decimal point in range $[0,1]$ with a number in
range $[0,1000]$ just multiplying by $1000$).

Then the number of possible values for $x_1$ less than $\frac{1}{3}$
is $334$ (because we can make a bijection between a number with three
digits after the decimal point in range $[0,\frac{1}{3}]$ with a
number in range $[0,333]$ just multiplying by $1000$).

Then we have that:

\begin{equation*}
    \begin{split}
        P(x_1 < \frac{1}{3}) & = \frac{334}{1001} \\
    \end{split}
\end{equation*}

To calculate $P(x_4 < \frac{1}{3})$ we can notice that $x_4$ has
the same distribution as $x_1$, so we have that:

\begin{equation*}
    \begin{split}
        P(x_4 < \frac{1}{3}) & = \frac{334}{1001} \\
    \end{split}
\end{equation*}

Then we have that:

\begin{equation*}
    \begin{split}
        P(\max(x_1, x_4) < \frac{1}{3}) & = P(x_1 < \frac{1}{3}) \cdot P(x_4 < \frac{1}{3}) \\
                                        & = \frac{334}{1001} \cdot \frac{334}{1001}         \\
                                        & = \frac{334^2}{1001^2}                            \\
                                        & = \frac{111556}{1002001}                          \\
                                        & = 0.11133322222
    \end{split}
\end{equation*}