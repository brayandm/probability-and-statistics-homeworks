\section*{Problem 3.4}
\addcontentsline{toc}{section}{Problem 3.4}

\textbf{Statement:}

Consider the random uniform number generator \texttt{rand} \textbf{with
    only 3 digits precision}, i.e., we consider equiprobable values
\( 0 \leq x_k \leq 1 \) with only three digits after the decimal
point. Assume we get from it the sequence of values \( x_1, x_2, x_3, x_4 \).

\begin{enumerate}
    \item Find the probability that \( \max(x_1, x_4) < \frac{1}{3} \).
    \item Find the probability that \( \min(x_1, x_2) < \frac{1}{3} \).
    \item Find the probability that \( \max(x_1, x_3) \) is in \( \left[ \frac{1}{4}, \frac{3}{4} \right] \).
    \item Find the probability that exactly two random values out of four are in \( \left[0, \frac{1}{3} \right] \).
\end{enumerate}

\noindent\textbf{Solution:}

\textbf{1)} Let's notice that $x_1$ and $x_4$ are independent
random variables, and they are both uniformly distributed in $[0,1]$
with only three digits after the decimal point.

So we have that $\max(x_1, x_4)$ should be less than $\frac{1}{3}$.
Then $x_1$ and $x_4$ should be both less than $\frac{1}{3}$, because
if one of them is greater or equal than $\frac{1}{3}$, then the maximum will
be greater or equal than $\frac{1}{3}$.

Then because $x_1$ and $x_4$ are independent, we have can calculate
sepately the probability that $x_1$ is less than $\frac{1}{3}$ and the
probability that $x_4$ is less than $\frac{1}{3}$, and then multiply them
together.

In other words, we have that:

\begin{equation*}
    \begin{split}
        P(\max(x_1, x_4) < \frac{1}{3}) & = P(x_1 < \frac{1}{3} \cap x_4 < \frac{1}{3})     \\
                                        & = P(x_1 < \frac{1}{3}) \cdot P(x_4 < \frac{1}{3}) \\
    \end{split}
\end{equation*}

To calculate $P(x_1 < \frac{1}{3})$ we can notice that $x_1$ is
uniformly distributed in $[0,1]$ with only three digits after the
decimal point. Then the total number of possible values for $x_1$ is
$1001$ (because we can make a bijection between a number with three
digits after the decimal point in range $[0,1]$ with a number in
range $[0,1000]$ just multiplying by $1000$).

Then the number of possible values for $x_1$ less than $\frac{1}{3}$
is $334$ (because we can make a bijection between a number with three
digits after the decimal point in range $[0,\frac{1}{3}]$ with a
number in range $[0,333]$ just multiplying by $1000$).

Then we have that:

\begin{equation*}
    \begin{split}
        P(x_1 < \frac{1}{3}) & = \frac{334}{1001} \\
    \end{split}
\end{equation*}

To calculate $P(x_4 < \frac{1}{3})$ we can notice that $x_4$ has
the same distribution as $x_1$, so we have that:

\begin{equation*}
    \begin{split}
        P(x_4 < \frac{1}{3}) & = \frac{334}{1001} \\
    \end{split}
\end{equation*}

Then we have that:

\begin{equation*}
    \begin{split}
        P(\max(x_1, x_4) < \frac{1}{3}) & = P(x_1 < \frac{1}{3}) \cdot P(x_4 < \frac{1}{3}) \\
                                        & = \frac{334}{1001} \cdot \frac{334}{1001}         \\
                                        & = \frac{334^2}{1001^2}                            \\
                                        & = \frac{111556}{1002001}                          \\
                                        & = 0.11133322222
    \end{split}
\end{equation*}

\textbf{2)} Let's notice that $x_1$ and $x_2$ are independent
random variables, and they are both uniformly distributed in $[0,1]$
with only three digits after the decimal point.

So we have that $\min(x_1, x_2)$ should be less than $\frac{1}{3}$.
But let's do the reverse, let's calculate the probability that
$\min(x_1, x_2)$ is greater or equal than $\frac{1}{3}$, and then
substract it from $1$.

Then $x_1$ and $x_2$ should be both greater or equal than
$\frac{1}{3}$, because if one of them is less than $\frac{1}{3}$, then
the minimum will be less than $\frac{1}{3}$.

Then because $x_1$ and $x_2$ are independent, we have can calculate
sepately the probability that $x_1$ is greater or equal than
$\frac{1}{3}$ and the probability that $x_2$ is greater or equal than
$\frac{1}{3}$, and then multiply them together.

In other words, we have that:

\begin{equation*}
    \begin{split}
        P(\min(x_1, x_2) \geq \frac{1}{3}) & = P(x_1 \geq \frac{1}{3} \cap x_2 \geq \frac{1}{3})     \\
                                           & = P(x_1 \geq \frac{1}{3}) \cdot P(x_2 \geq \frac{1}{3}) \\
    \end{split}
\end{equation*}

To calculate $P(x_1 \geq \frac{1}{3})$ we can notice that $x_1$ is
uniformly distributed in $[0,1]$ with only three digits after the
decimal point. Then the total number of possible values for $x_1$ is
$1001$ (because we can make a bijection between a number with three
digits after the decimal point in range $[0,1]$ with a number in
range $[0,1000]$ just multiplying by $1000$).

Then the number of possible values for $x_1$ greater or equal than
$\frac{1}{3}$ is $667$ (because we can make a bijection between a number with three
digits after the decimal point in range $[\frac{1}{3},1]$ with a
number in range $[334,1000]$ just multiplying by $1000$)

Then we have that:

\begin{equation*}
    \begin{split}
        P(x_1 \geq \frac{1}{3}) & = \frac{667}{1001} \\
    \end{split}
\end{equation*}

To calculate $P(x_2 \geq \frac{1}{3})$ we can notice that $x_2$ has
the same distribution as $x_1$, so we have that:

\begin{equation*}
    \begin{split}
        P(x_2 \geq \frac{1}{3}) & = \frac{667}{1001} \\
    \end{split}
\end{equation*}

Then we have that:

\begin{equation*}
    \begin{split}
        P(\min(x_1, x_2) \geq \frac{1}{3}) & = P(x_1 \geq \frac{1}{3}) \cdot P(x_2 \geq \frac{1}{3}) \\
                                           & = \frac{667}{1001} \cdot \frac{667}{1001}               \\
                                           & = \frac{667^2}{1001^2}                                  \\
                                           & = \frac{444889}{1002001}                                \\
                                           & = 0.44400055489
    \end{split}
\end{equation*}

Then we have that:

\begin{equation*}
    \begin{split}
        P(\min(x_1, x_2) < \frac{1}{3}) & = 1 - P(\min(x_1, x_2) \geq \frac{1}{3}) \\
                                        & = 1 - 0.44400055489                      \\
                                        & = 0.55599944511
    \end{split}
\end{equation*}

\textbf{3)} Let's notice that $x_1$ and $x_3$ are independent
random variables, and they are both uniformly distributed in $[0,1]$
with only three digits after the decimal point.

So we have that $\max(x_1, x_3)$ should be in $[ \frac{1}{4}, \frac{3}{4} ]$.
But notice that we can decompose the problem by calculating the
probability that $\max(x_1, x_3)$ is less or equal than $\frac{3}{4}$ and
substracting the probability that $\max(x_1, x_3)$ is less than
$\frac{1}{4}$.

Then we have that:

\begin{equation*}
    \begin{split}
        P(\frac{1}{4}\leq\max(x_1, x_3)\leq\frac{3}{4}) & = P(\max(x_1, x_3) \leq \frac{3}{4}) - P(\max(x_1, x_3) < \frac{1}{4}) \\
    \end{split}
\end{equation*}

Then by previous results we have that:

\begin{equation*}
    \begin{split}
        P(\frac{1}{4}\leq\max(x_1, x_3)\leq\frac{3}{4}) & = P(\max(x_1, x_3) \leq \frac{3}{4}) - P(\max(x_1, x_3) < \frac{1}{4})                                    \\
                                                        & = P(x_1 \leq \frac{3}{4} \cap x_3 \leq \frac{3}{4}) - P(x_1 < \frac{1}{4} \cap x_3 < \frac{1}{4})         \\
                                                        & = P(x_1 \leq \frac{3}{4}) \cdot P(x_3 \leq \frac{3}{4}) - P(x_1 < \frac{1}{4}) \cdot P(x_3 < \frac{1}{4}) \\
    \end{split}
\end{equation*}

To calculate $P(x_1 \leq \frac{3}{4})$ we can notice that $x_1$ is
uniformly distributed in $[0,1]$ with only three digits after the
decimal point. Then the total number of possible values for $x_1$ is
$1001$ (because we can make a bijection between a number with three

Then the number of possible values for $x_1$ less or equal than
$\frac{3}{4}$ is $751$ (because we can make a bijection between a number with three
digits after the decimal point in range $[0,\frac{3}{4}]$ with a
number in range $[0,750]$ just multiplying by $1000$).

Then we have that:

\begin{equation*}
    \begin{split}
        P(x_1 \leq \frac{3}{4}) & = \frac{751}{1001} \\
    \end{split}
\end{equation*}

To calculate $P(x_3 \leq \frac{3}{4})$ we can notice that $x_3$ has
the same distribution as $x_1$, so we have that:

\begin{equation*}
    \begin{split}
        P(x_3 \leq \frac{3}{4}) & = \frac{751}{1001} \\
    \end{split}
\end{equation*}

To calculate $P(x_1 < \frac{1}{4})$ we can notice that $x_1$ is
uniformly distributed in $[0,1]$ with only three digits after the
decimal point. Then the total number of possible values for $x_1$ is
$1001$ (because we can make a bijection between a number with three
digits after the decimal point in range $[0,1]$ with a number in
range $[0,1000]$ just multiplying by $1000$).

Then the number of possible values for $x_1$ less than $\frac{1}{4}$
is $250$ (because we can make a bijection between a number with three
digits after the decimal point in range $[0,\frac{1}{4})$ with a
                number in range $[0,249]$ just multiplying by $1000$).

                Then we have that:

                \begin{equation*}
                    \begin{split}
                        P(x_1 < \frac{1}{4}) & = \frac{250}{1001} \\
                    \end{split}
                \end{equation*}

                To calculate $P(x_3 < \frac{1}{4})$ we can notice that $x_3$ has
                the same distribution as $x_1$, so we have that:

                \begin{equation*}
                    \begin{split}
                        P(x_3 < \frac{1}{4}) & = \frac{250}{1001} \\
                    \end{split}
                \end{equation*}

                Then we have that:

                \begin{equation*}
                    \begin{split}
                        P(\frac{1}{4}\leq\max(x_1, x_3)\leq\frac{3}{4}) & = P(x_1 \leq \frac{3}{4}) \cdot P(x_3 \leq \frac{3}{4}) - P(x_1 < \frac{1}{4}) \cdot P(x_3 < \frac{1}{4}) \\
                                                                        & = \frac{751}{1001} \cdot \frac{751}{1001} - \frac{250}{1001} \cdot \frac{250}{1001}                       \\
                                                                        & = \frac{751^2}{1001^2} - \frac{250^2}{1001^2}                                                             \\
                                                                        & = \frac{751^2 - 250^2}{1001^2}                                                                            \\
                                                                        & = \frac{564001 - 62500}{1002001}                                                                          \\
                                                                        & = \frac{501501}{1002001}                                                                                  \\
                                                                        & = 0.5004995005
                    \end{split}
                \end{equation*}

                \textbf{4)} Let's notice that $x_1$, $x_2$, $x_3$ and $x_4$ are independent
                random variables, and they are all uniformly distributed in $[0,1]$
                with only three digits after the decimal point.

                Also notice that there are $6$ ways to choose exactly two random
                values out of four (just using combinations n over k), so we only need to calculate the probability of
                one of them and then multiply it by $6$.

                Let's select $x_1$ and $x_2$ to be in $[0, \frac{1}{3}]$ and $x_3$ and
            $x_4$ to be in $(\frac{1}{3}, 1]$.

Then we have that:

\begin{equation*}
    \begin{split}
        P(0 \leq x_1, x_2 \leq \frac{1}{3} \cap \frac{1}{3} < x_3, x_4 \leq1) & =P(0 \leq x_1, x_2 \leq \frac{1}{3}) \cdot P(\frac{1}{3} < x_3, x_4 \leq1)                                                             \\
                                                                              & =P(0 \leq x_1 \leq \frac{1}{3}) \cdot P(0 \leq x_2 \leq \frac{1}{3}) \cdot P(\frac{1}{3} < x_3 \leq1) \cdot P(\frac{1}{3} < x_4 \leq1) \\
                                                                              & =P(0 \leq x_1 \leq \frac{1}{3}) \cdot P(0 \leq x_2 \leq \frac{1}{3}) \cdot P(\frac{1}{3} < x_3 \leq1) \cdot P(\frac{1}{3} < x_4 \leq1) \\
    \end{split}
\end{equation*}

To calculate $P(0 \leq x_1 \leq \frac{1}{3})$ we can notice that $x_1$ is
uniformly distributed in $[0,1]$ with only three digits after the
decimal point. Then the total number of possible values for $x_1$ is
$1001$ (because we can make a bijection between a number with three
digits after the decimal point in range $[0,1]$ with a number in
range $[0,1000]$ just multiplying by $1000$).

Then the number of possible values for $x_1$ in $[0, \frac{1}{3}]$ is $334$ (because we can make a bijection between a number with three
digits after the decimal point in range $[0,\frac{1}{3}]$ with a
number in range $[0,333]$ just multiplying by $1000$).

Then we have that:

\begin{equation*}
    \begin{split}
        P(0 \leq x_1 \leq \frac{1}{3}) & = \frac{334}{1001} \\
    \end{split}
\end{equation*}

To calculate $P(0 \leq x_2 \leq \frac{1}{3})$ we can notice that $x_2$ has
the same distribution as $x_1$, so we have that:

\begin{equation*}
    \begin{split}
        P(0 \leq x_2 \leq \frac{1}{3}) & = \frac{334}{1001} \\
    \end{split}
\end{equation*}

To calculate $P(\frac{1}{3} < x_3 \leq1)$ we can notice that $x_3$ is
uniformly distributed in $[0,1]$ with only three digits after the
decimal point. Then the total number of possible values for $x_3$ is
$1001$ (because we can make a bijection between a number with three
digits after the decimal point in range $[0,1]$ with a number in
range $[0,1000]$ just multiplying by $1000$).

Then the number of possible values for $x_3$ in $(\frac{1}{3}, 1]$ is $667$ (because we can make a bijection between a number with three
digits after the decimal point in range $(\frac{1}{3}, 1]$ with a
number in range $[334,1000]$ just multiplying by $1000$).

Then we have that:

\begin{equation*}
    \begin{split}
        P(\frac{1}{3} < x_3 \leq1) & = \frac{667}{1001} \\
    \end{split}
\end{equation*}

To calculate $P(\frac{1}{3} < x_4 \leq1)$ we can notice that $x_4$ has
the same distribution as $x_3$, so we have that:

\begin{equation*}
    \begin{split}
        P(\frac{1}{3} < x_4 \leq1) & = \frac{667}{1001} \\
    \end{split}
\end{equation*}

Then we have that:

\begin{equation*}
    \begin{split}
        P(0 \leq x_1, x_2 \leq \frac{1}{3} \cap \frac{1}{3} < x_3, x_4 \leq1) & =P(0 \leq x_1 \leq \frac{1}{3}) \cdot P(0 \leq x_2 \leq \frac{1}{3}) \cdot P(\frac{1}{3} < x_3 \leq1) \cdot P(\frac{1}{3} < x_4 \leq1) \\
                                                                              & =\frac{334}{1001} \cdot \frac{334}{1001} \cdot \frac{667}{1001} \cdot \frac{667}{1001}                                                 \\
                                                                              & =\frac{334^2 \cdot 667^2}{1001^4}                                                                                                      \\
                                                                              & =\frac{111556\cdot 444889}{1002001^2}                                                                                                  \\
                                                                              & =\frac{49630037284}{1004006004001}                                                                                                     \\
                                                                              & =0.04943201244
    \end{split}
\end{equation*}

Then we have that:

\begin{equation*}
    \begin{split}
        P(\text{exactly two random values out of four are in } [0, \frac{1}{3}]) & = 6 \cdot P(0 \leq x_1, x_2 \leq \frac{1}{3} \cap \frac{1}{3} < x_3, x_4 \leq1) \\
                                                                                 & = 6 \cdot 0.04943201244                                                         \\
                                                                                 & = 0.29659207464
    \end{split}
\end{equation*}