\section*{Problem 5.5}
\addcontentsline{toc}{section}{Problem 5.5}

\subsection*{Statement: 5.5}
Let $\xi, \eta : \Omega \rightarrow \mathbb{R}$.

1. Is it true that if $\xi$, $\eta$ are independent, then $\xi^2$ $\eta^2$ are also independent?

2. Is it true that if $\xi$, $\eta$ are independent, then for an arbitrary function $f : \mathbb{R} \rightarrow \mathbb{R}$, random variables $f(\xi)$ $f(\eta)$ are also independent?

3. Is it true that if $\xi$, $\eta$ are dependent, then $\xi^2$ $\eta^2$ are necessarily dependent?


\subsection*{Solution: 5.5.1}

\textbf{Proof:}

\textit{Given:}
\begin{itemize}
    \item \(\xi\) and \(\eta\) are independent random variables.
    \item Independence is defined as: For all values \( x \) and \( y \),
          $$ P(\xi = x \text{ and } \eta = y) = P(\xi = x) \cdot P(\eta = y) $$
\end{itemize}

\textit{To Prove:}
\begin{itemize}
    \item \(\xi^2\) and \(\eta^2\) are independent.
\end{itemize}

\textit{Proof:}
\begin{enumerate}
    \item \textbf{Decomposing the Events for Squares}:
          \begin{itemize}
              \item The event \(\xi^2 = a\) occurs if and only if \(\xi = \sqrt{a}\) or \(\xi = -\sqrt{a}\), assuming \(a\) is a non-negative number. Similarly, \(\eta^2 = b\) occurs if and only if \(\eta = \sqrt{b}\) or \(\eta = -\sqrt{b}\).
          \end{itemize}
    \item \textbf{Using Independence of \(\xi\) and \(\eta\)}:
          \begin{itemize}
              \item Since \(\xi\) and \(\eta\) are independent, the event \(\xi = \sqrt{a}\) or \(\xi = -\sqrt{a}\) is independent of the event \(\eta = \sqrt{b}\) or \(\eta = -\sqrt{b}\).
          \end{itemize}
    \item \textbf{Probability of Union of Independent Events}:
          \begin{itemize}
              \item For two independent events \(X\) and \(Y\), the probability of both occurring is the product of their individual probabilities: \(P(X \text{ and } Y) = P(X) \cdot P(Y)\).
          \end{itemize}
    \item \textbf{Applying to Squares}:
          \begin{itemize}
              \item Applying this to our case, we get:
                    \[ P(\xi^2 = a \text{ and } \eta^2 = b) = P((\xi = \sqrt{a} \text{ or } \xi = -\sqrt{a}) \text{ and } (\eta = \sqrt{b} \text{ or } \eta = -\sqrt{b})) \]
              \item Since the events \(\xi = \sqrt{a}\) or \(\xi = -\sqrt{a}\) and \(\eta = \sqrt{b}\) or \(\eta = -\sqrt{b}\) are independent, this simplifies to:
                    \[ P(\xi^2 = a \text{ and } \eta^2 = b) = P(\xi = \sqrt{a} \text{ or } \xi = -\sqrt{a}) \cdot P(\eta = \sqrt{b} \text{ or } \eta = -\sqrt{b}) \]
              \item So replacing back we get:
                    \[ P(\xi^2 = a \text{ and } \eta^2 = b) = P(\xi^2 = a) \cdot P(\eta^2 = b) \]
          \end{itemize}
\end{enumerate}

\textit{Conclusion:}
\begin{itemize}
    \item Since the above relationship holds for all values \( a \) and \( b \), it follows that \(\xi^2\) and \(\eta^2\) are independent.
\end{itemize}

Then the statement is \textbf{true}.

\subsection*{Solution: 5.5.2}

\textbf{Proof:}

\textit{Given:}
\begin{itemize}
    \item \(\xi\) and \(\eta\) are independent random variables.
    \item A function \(f : \mathbb{R} \rightarrow \mathbb{R}\).
\end{itemize}

\textit{To Prove:}
\begin{itemize}
    \item The random variables \(f(\xi)\) and \(f(\eta)\) are independent.
\end{itemize}

\textit{Proof:}
\begin{enumerate}
    \item \textbf{Definition of Independence:}
          \begin{itemize}
              \item Two random variables \(X\) and \(Y\) are independent if for every pair of Borel sets \(A\) and \(B\), \(P(X \in A \text{ and } Y \in B) = P(X \in A) \cdot P(Y \in B)\).
          \end{itemize}

    \item \textbf{Applying to \(f(\xi)\) and \(f(\eta)\):}
          \begin{itemize}
              \item Given that \(\xi\) and \(\eta\) are independent, for any Borel sets \(A\) and \(B\), we have \(P(\xi \in A \text{ and } \eta \in B) = P(\xi \in A) \cdot P(\eta \in B)\).
          \end{itemize}

    \item \textbf{Transformation by \(f\):}
          \begin{itemize}
              \item Consider the transformed variables \(f(\xi)\) and \(f(\eta)\). For any Borel sets \(C\) and \(D\) in the range of \(f\), we can find corresponding sets \(A\) and \(B\) in the domain of \(f\) such that \(f^{-1}(C) = A\) and \(f^{-1}(D) = B\).
          \end{itemize}

    \item \textbf{Independence of Transformed Variables:}
          \begin{itemize}
              \item Since \(\xi\) and \(\eta\) are independent, we have:
                    \[ P(\xi \in f^{-1}(C) \text{ and } \eta \in f^{-1}(D)) = P(\xi \in f^{-1}(C)) \cdot P(\eta \in f^{-1}(D)) \]
              \item This can be rewritten as:
                    \[ P(f(\xi) \in C \text{ and } f(\eta) \in D) = P(f(\xi) \in C) \cdot P(f(\eta) \in D) \]
              \item This shows that \(f(\xi)\) and \(f(\eta)\) are independent.
          \end{itemize}
\end{enumerate}

\textit{Conclusion:}
\begin{itemize}
    \item If \(\xi\) and \(\eta\) are independent random variables, then for an arbitrary function \(f : \mathbb{R} \rightarrow \mathbb{R}\), the random variables \(f(\xi)\) and \(f(\eta)\) are also independent.
\end{itemize}

Then the statement is \textbf{true}.

\subsection*{Solution: 5.5.3}

To find a counterexample to the statement "If $\xi$ and $\eta$ are dependent, then $\xi^2$ and $\eta^2$ are necessarily dependent," we consider the following:

Let $\xi$ be a random variable that takes values 1 and -1 with equal probability, i.e.,
\[ P(\xi = 1) = P(\xi = -1) = 0.5. \]

Define $\eta$ as $\eta = \xi$. Thus, $\eta$ is completely dependent on $\xi$.

Now, consider $\xi^2$ and $\eta^2$:
\begin{itemize}
    \item $\xi^2$ will always be 1, since squaring either 1 or -1 results in 1.
    \item Similarly, $\eta^2$ will always be 1, since $\eta$ is equal to $\xi$.
\end{itemize}

Therefore, $\xi^2$ and $\eta^2$ are both constant random variables. Constant random variables are independent of any other random variable, including each other, because the occurrence of any particular value of one does not affect the probability distribution of the other.

Hence, $\xi^2$ and $\eta^2$ are independent, even though $\xi$ and $\eta$ are dependent. This serves as a counterexample to the original statement.

Then the statement is \textbf{false}.