\section*{Solution Problem 3}

To find the distribution function and probability density function (PDF)
of \(\mu = \beta / \alpha\), where \(\alpha\) and \(\beta\) are independent
random variables uniformly distributed over \([1, 2]\) and \([0, 1]\) respectively,
we can make a geometric approach. Notice that the total area of the square
$[0, 1] \times [1, 2]$ is 1 so the probability of $\alpha$ and $\beta$ being
in this square is 1.

Note that $\beta = \mu \alpha$

To find the distribution function $F_{\mu}(x)$, we can use the following:

\[ F_{\mu}(x) = P(\mu \leq x) = P(\beta / \alpha \leq x) = P(\beta \leq x \alpha) \]

Then to be $\beta \leq x \alpha$, $\beta$ must be in the interval $[0, x \alpha]$
for $\alpha \in [1, 2]$.

Geometrically, $x \alpha$ is the line $y = x \alpha$ and we can notice that
there are two main cases:

1) $0 < x \leq \frac{1}{2}$: In this case, the line $y = x \alpha$ is going up
until intersecting upper right corner of the square $[0, 1] \times [1, 2]$.

to calculate this probability, we can integrate the area under the line
$y = x \alpha$ from $1$ to $2$:

\[ F_{\mu}(x) = \int_{1}^{2} x \alpha d\alpha = \frac{1}{2} x \alpha^2 \Big|_{1}^{2} = \frac{3}{2}x \]

2) $\frac{1}{2} < x \leq 1$: In this case, the line $y = x \alpha$ is going up
until intersecting upper left corner of the square $[0, 1] \times [1, 2]$.

In this case we can notice that the complementary of the area under
the line $y = x \alpha$ is the area of the triangle with vertices made
by the upper left corner of the square $[0, 1] \times [1, 2]$, the intersection
of the line $y = x \alpha$ with left side of the square, and
the intersection of the line $y = x \alpha$ with the top side of the square.

This vertices are $(1, 1)$, $(1, x)$, and $(\frac{1}{x}, 1)$ respectively.

The area is calculated by formula of right triangle:

The base is $1 - x$ and the height is $\frac{1}{x} - 1$

So the area is $\frac{1}{2} (1 - x) (\frac{1}{x} - 1) = \frac{1}{2} (\frac{1}{x}-1-1+x)
    = \frac{1}{2} (\frac{1}{x} - 2 + x) = \frac{1}{2x} - 1 + x/2$

Then the distribution function is:

\[ F_{\mu}(x) = \begin{cases}
        0                      & \text{if } x \leq 0               \\
        \frac{3}{2}x           & \text{if } 0 < x \leq \frac{1}{2} \\
        \frac{1}{2x} - 1 + x/2 & \text{if } \frac{1}{2} < x \leq 1 \\
        1                      & \text{if } x > 1
    \end{cases} \]

To find the probability density function $f_{\mu}(x)$, we can derive the
distribution function:

\[ f_{\mu}(x) = \begin{cases}
        \frac{3}{2}           & \text{if } 0 < x \leq \frac{1}{2} \\
        -\frac{1}{2x^2} + 1/2 & \text{if } \frac{1}{2} < x \leq 1 \\
        0                     & \text{otherwise}
    \end{cases} \]
