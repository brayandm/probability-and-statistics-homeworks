\section*{Solution Problem 1}

To find the distribution function and probability density function (PDF) of \(\xi = \alpha + \beta\), where \(\alpha\) and \(\beta\) are independent random variables uniformly distributed over \([1, 2]\) and \([0, 1]\) respectively, we'll follow these steps:

1. Understand the Distributions of \(\alpha\) and \(\beta\):

- \(\alpha \sim U(1, 2)\): This means \(\alpha\) is uniformly distributed over the interval [1, 2]. Its PDF is \(f_{\alpha}(\alpha) = 1\) for \(\alpha \in [1, 2]\) and 0 otherwise.

- \(\beta \sim U(0, 1)\): This means \(\beta\) is uniformly distributed over the interval [0, 1]. Its PDF is \(f_{\beta}(\beta) = 1\) for \(\beta \in [0, 1]\) and 0 otherwise.

2. Find the PDF of \(\xi = \alpha + \beta\):

The PDF of the sum of two independent random variables can be found by convoluting their individual PDFs. The convolution of two functions \(f\) and \(g\) is given by:

\[ (f * g)(t) = \int_{-\infty}^{\infty} f(\tau)g(t - \tau) d\tau \]

Applying this to our situation:

\[ f_{\xi}(\xi) = (f_{\alpha} * f_{\beta})(\xi) = \int_{-\infty}^{\infty} f_{\alpha}(\tau)f_{\beta}(\xi - \tau) d\tau \]

Given the specific intervals where \(f_{\alpha}\) and \(f_{\beta}\) are nonzero, this integral simplifies to:

\[ f_{\xi}(\xi) = \int_{1}^{2} f_{\beta}(\xi - \tau) d\tau \]

Because \(f_{\beta}(\beta) = 1\) for \(\beta \in [0, 1]\), this integral becomes non-zero only when \(\xi - \tau \in [0, 1]\).

3. Compute the PDF \(f_{\xi}(\xi)\):

- For \(\xi \in [1, 2]\) (where \(\alpha + \beta\) is just starting to get contributions), the integration limits for \(\tau\) are from max(1, \(\xi - 1\)) to \(\xi\). So, \(f_{\xi}(\xi) = \xi - 1\).

- For \(\xi \in [2, 3]\) (where \(\alpha + \beta\) gets the full range of contributions), the integration limits for \(\tau\) are from max(1, \(\xi - 1\)) to 2. So, \(f_{\xi}(\xi) = 3 - \xi\).

- For the rest of the values of \(\xi\), \(f_{\xi}(\xi) = 0\) (since the sum cannot exceed 3 and not be less than 1).

4. Find the Distribution Function \(F_{\xi}(\xi)\):

The distribution function is the integral of the PDF from \(-\infty\) to \(\xi\). For our case:

- For \(\xi < 1\), \(F_{\xi}(\xi) = 0\) (since both \(\alpha\) and \(\beta\) are greater than or equal to 1).

- For \(\xi \in [1, 2]\), \(F_{\xi}(\xi) = \int_{1}^{\xi} (\tau - 1) d\tau = (\frac{1}{2}\tau^2 - \tau) \Big|_{1}^{\xi} = \frac{1}{2}\xi^2 - \xi + \frac{1}{2}\).

- For \(\xi \in [2, 3]\), \(F_{\xi}(\xi) = \int_{1}^{2} (\tau - 1) d\tau + \int_{2}^{\xi} (3 - \tau) d\tau  = (\frac{1}{2}\tau^2 - \tau) \Big|_{1}^{2} + (3\tau - \frac{1}{2}\tau^2) \Big|_{2}^{\xi} = \frac{-\xi^2}{2} + 3\xi - \frac{7}{2} \).

- For \(\xi > 3\), \(F_{\xi}(\xi) = 1\) (since the sum cannot exceed 3).