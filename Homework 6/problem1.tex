\section*{Problem 1}
\addcontentsline{toc}{section}{Problem 1}

Let we have two r.v. as mappings from $[0,1] \to \mathbb{R}$, $\xi$ and $\eta$.

Let there are explicit formulas for them as $x^3$ and $x^2$.

% 1) Find the expectations $E(\xi)$, $E(\eta)$?

% 2) Whether $\xi$ and $\eta$ are dependent?

% 3) Find the explicit formulas for their distributions functions.

% 4) Use the formula to compute expectations by formula of distribution function and compare the results

\subsection*{Find the expectations $E(\xi)$, $E(\eta)$?}

We have that the expectation of a r.v. is defined as:

\begin{equation*}
    E(\xi) = \int_{-\infty}^{\infty} g(x) f(x) dx
\end{equation*}

Where $g(x)$ is the function of the r.v. and $f(x)$ is the probability density function.

In our case, we have that $g(x) = x^3$ and $f(x) = 1$ (because the r.v. is uniformly distributed).

And because the range of the r.v. is $[0,1]$, we have that the expectation is:

\begin{equation*}
    E(\xi) = \int_{0}^{1} x^3 dx = \frac{x^4}{4} \Big|_{0}^{1} = \frac{1}{4}
\end{equation*}

And for $\eta$:

\begin{equation*}
    E(\eta) = \int_{0}^{1} x^2 dx = \frac{x^3}{3} \Big|_{0}^{1} = \frac{1}{3}
\end{equation*}

\subsection*{Whether $\xi$ and $\eta$ are dependent?}

We can check this dependency by counterexample.

Let take the set H = [0, 1/2]

For $\xi$ and $\eta$ to be independent, we need that:

\begin{equation*}
    P(\xi \in H, \eta \in H) = P(\xi \in H) P(\eta \in H)
\end{equation*}

because in other case, they are dependent, due to the definition
of independence of r.v.

Now let's use the inverse image of $\xi$ and $\eta$:

\begin{equation*}
    P(x \in \xi^{-1}(H), x \in \eta^{-1}(H)) = P(x \in \xi^{-1}(H)) P(x \in \eta^{-1}(H))
\end{equation*}

In the left side of the equation, we have that $\xi^{-1}(H) = [0, \sqrt[3]{1/2}]$ and $\eta^{-1}(H) = [0, \sqrt{1/2}]$.

So:

\begin{equation*}
    P(x \in \xi^{-1}(H), x \in \eta^{-1}(H)) = P(x \in [0, \sqrt[3]{1/2}] \cap [0, \sqrt{1/2}]) = P(x \in [0, \sqrt[3]{1/2}]) = \sqrt[3]{1/2}
\end{equation*}

In the right side of the equation, we have that $P(x \in \xi^{-1}(H)) = P(x \in [0, \sqrt[3]{1/2}]) = \sqrt[3]{1/2}$ and $P(x \in \eta^{-1}(H)) = P(x \in [0, \sqrt{1/2}]) = \sqrt{1/2}$.

So:

\begin{equation*}
    P(x \in \xi^{-1}(H)) P(x \in \eta^{-1}(H)) = \sqrt[3]{1/2} \sqrt{1/2} = \sqrt[6]{1/32}
\end{equation*}

So we have that:

\begin{equation*}
    \sqrt[3]{1/2} \neq \sqrt[6]{1/32}
\end{equation*}

So $\xi$ and $\eta$ are dependent.

\subsection*{Find the explicit formulas for their distributions functions.}

Let's find the distribution function for $\xi$:

From the definition of distribution function, we have that:

\begin{equation*}
    F_{\xi}(w) = P(\xi \leq w)
\end{equation*}

And because $\xi = x^3$, and $x \in [0,1]$, we have that:

\begin{equation*}
    F_{\xi}(w) = P(x^3 \leq w) = P(x \leq \sqrt[3]{w})
\end{equation*}

Then we have three cases:

for $w < 0$, $F_{\xi}(w) = 0$

for $0 \leq w \leq 1$, $F_{\xi}(w) = P(x \leq \sqrt[3]{w}) = \sqrt[3]{w}$

for $w > 1$, $F_{\xi}(w) = 1$

So we have that:

\begin{equation*}
    F_{\xi}(w) = \begin{cases}
        0           & \text{if } w < 0           \\
        \sqrt[3]{w} & \text{if } 0 \leq w \leq 1 \\
        1           & \text{if } w > 1
    \end{cases}
\end{equation*}

Let's find the distribution function for $\eta$:

From the definition of distribution function, we have that:

\begin{equation*}
    F_{\eta}(w) = P(\eta \leq w)
\end{equation*}

And because $\eta = x^2$, and $x \in [0,1]$, we have that:

\begin{equation*}
    F_{\eta}(w) = P(x^2 \leq w) = P(x \leq \sqrt{w})
\end{equation*}

Then we have three cases:

for $w < 0$, $F_{\eta}(w) = 0$

for $0 \leq w \leq 1$, $F_{\eta}(w) = P(x \leq \sqrt{w}) = \sqrt{w}$

for $w > 1$, $F_{\eta}(w) = 1$

So we have that:

\begin{equation*}
    F_{\eta}(w) = \begin{cases}
        0        & \text{if } w < 0           \\
        \sqrt{w} & \text{if } 0 \leq w \leq 1 \\
        1        & \text{if } w > 1
    \end{cases}
\end{equation*}

\subsection*{Use the formula to compute expectations by formula of distribution function and compare the results}

We have that the expectation of a r.v. is defined as:

\begin{equation*}
    E(\xi) = \int_{-\infty}^{\infty} x f_\xi(x) dx
\end{equation*}

Where $f_\xi(x)$ is the probability density function.

So let's find the probability density function for $\xi$ and $\eta$:

We have that the density function is the derivative of the distribution function.

So for $\xi$:

\begin{equation*}
    f_\xi(w) = \frac{d}{dw} F_\xi(w) = \frac{d}{dw} \begin{cases}
        0           & \text{if } w < 0           \\
        \sqrt[3]{w} & \text{if } 0 \leq w \leq 1 \\
        1           & \text{if } w > 1
    \end{cases} = \begin{cases}
        0                            & \text{if } w < 0           \\
        \frac{1}{3} w^{-\frac{2}{3}} & \text{if } 0 \leq w \leq 1 \\
        0                            & \text{if } w > 1
    \end{cases}
\end{equation*}

And for $\eta$:

\begin{equation*}
    f_\eta(w) = \frac{d}{dw} F_\eta(w) = \frac{d}{dw} \begin{cases}
        0        & \text{if } w < 0           \\
        \sqrt{w} & \text{if } 0 \leq w \leq 1 \\
        1        & \text{if } w > 1
    \end{cases} = \begin{cases}
        0                            & \text{if } w < 0           \\
        \frac{1}{2} w^{-\frac{1}{2}} & \text{if } 0 \leq w \leq 1 \\
        0                            & \text{if } w > 1
    \end{cases}
\end{equation*}

So now let's use the formula for the expectation:

for $\xi$:

\begin{equation*}
    E(\xi) = \int_{-\infty}^{\infty} x f_\xi(x) dx = \int_{0}^{1} x \frac{1}{3} x^{-\frac{2}{3}} dx = \frac{1}{3} \int_{0}^{1} x^{\frac{1}{3}} dx = \frac{1}{3} \frac{3}{4} x^{\frac{4}{3}} \Big|_{0}^{1} = \frac{1}{4}
\end{equation*}

And for $\eta$:

\begin{equation*}
    E(\eta) = \int_{-\infty}^{\infty} x f_\eta(x) dx = \int_{0}^{1} x \frac{1}{2} x^{-\frac{1}{2}} dx = \frac{1}{2} \int_{0}^{1} x^{\frac{1}{2}} dx = \frac{1}{2} \frac{2}{3} x^{\frac{3}{2}} \Big|_{0}^{1} = \frac{1}{3}
\end{equation*}

So we have that both results are the same as the ones we got in the first part of the problem.